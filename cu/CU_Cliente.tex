% \IUref{IUAdmPS}{Administrar Planta de Selección}
% \IUref{IUModPS}{Modificar Planta de Selección}
% \IUref{IUEliPS}{Eliminar Planta de Selección}

% 


% Copie este bloque por cada caso de uso:
%-------------------------------------- COMIENZA descripción del caso de uso.

%\begin{UseCase}[archivo de imágen]{UCX}{Nombre del Caso de uso}{
	\begin{UseCase}{CU17}{Inscribir a Seminario}{
		Ayudar a que los Estudiantes que están por terminar la carrera se puedan inscribir en un Seminario de titulación.
	}
		\UCitem{Versión}{0.1}
		\UCitem{Actor}{Recepcionista}
		\UCitem{Propósito}{Que la recepcionista pueda añadir un cliente nuevo al sistema.}
		\UCitem{Entradas}{Nombre, Apellido Paterno, Apellido Materno, CURP, Fecha de Nacimiento, Género, Correo, Teléfono, Tipo de persona, RFC, Dirección Física, Dirección Fiscal, Razón Social, Enfermedades Cardiovasculares, Enfermedades Respiratorias, Enfermedades Crónicas, Enfermedades Digestivas, Fracturas y Alergias.}
		\UCitem{Origen}{Teclado}
		\UCitem{Salidas}{Cliente registrado, datos del cliente, usuario registrado y datos del usuario.}
		\UCitem{Destino}{Pantalla.}
		\UCitem{Precondiciones}{La recepcionista puede añadir un nuevo cliente al sistema si y sólo si el aspirante a cliente es mayor de edad.}
		\UCitem{Postcondiciones}{La recepcionista habrá registrado al cliente en el sistema como un cliente nuevo con él mismo como usuario asociado, si el cliente cumplió con el requisito de ser mayor de edad.}
		\UCitem{Errores}{}
		\UCitem{Tipo}{Caso de uso primario}
		\UCitem{Observaciones}{}
		\UCitem{Autor}{Raúl Kevin Monteón Valdés.}
	\end{UseCase}

	\begin{UCtrayectoria}{Principal}
		\UCpaso 1. Despliega los clientes registrados, ordenados por fecha de registro vía la  \IUref{IU01}{Pantalla de listado de clientes registrados en el sistema}.
		
		\UCpaso[\UCactor] 2. Hace clic en el botón Agregar Cliente.
		
		\UCpaso 3. Despliega el formulario de datos principales del aspirante a usuario  \IUref{IU02}{Pantalla de registro de clientes y usuarios}.
		
		\UCpaso[\UCactor] 4. Introduce Nombre, Apellido Paterno, Apellido Materno, CURP, Fecha de Nacimiento, Género, Correo, Teléfono, Tipo de persona, RFC, Dirección Física, Dirección Fiscal y Razón Social del aspirante a cliente.
		
		\UCpaso 5. Verifica que el aspirante a cliente sea mayor de edad con base en su fecha de nacimiento por la regla  \BRref{BR001}{Determinar si un aspirante a cliente puede ser registrado en el sistema} \Trayref{A}.
		
		\UCpaso 6. Verifica que el CURP del aspirante a cliente sea válido con base en la regla  \BRref{BR002}{Determinar si un CURP es válido} \Trayref{B}.
		
		\UCpaso 7. Extrae los datos del CURP (iniciales de nombre y apellidos, fecha de nacimiento, entidad de nacimiento, género).
		
		\UCpaso 8. Verifica que el los datos del usuario coincidan con los datos extraídos del CURP.
		
		\UCpaso 9. Verifica que el correo introducido tiene un formato correcto con base en la regla  \BRref{BR003}{Determinar si un correo tiene el formato correcto} \Trayref{C}.
		
		\UCpaso 10. Verifica que teléfono introducido tiene un formato correcto con base en la regla  \BRref{BR004}{Determinar si un teléfono tiene el formato correcto} \Trayref{D}.
		
		\UCpaso 11. Verifica que el RFC introducido es congruente con el CURP introducido con base en la regla  \BRref{BR005}{Determinar si un RFC es congruente con un CURP} \Trayref{E}.
		
		\UCpaso 12. Despliega el formulario de datos médicos vía la \IUref{IU03}{Pantalla de registro de datos médicos}.
		
		\UCpaso[\UCactor] 13. Introduce las Enfermedades Cardiovasculares, Enfermedades Respiratorias, Enfermedades Crónicas, Enfermedades Digestivas, Fracturas y Alergias que padezca el usuario.		
		
		\UCpaso[\UCactor] 14. Confirma el registro del cliente.
		
		\UCpaso 15. Registra datos del cliente.
		
		\UCpaso 16. Notifica que se hizo un registro exitoso vía la \IUref{IU04}{Pantalla de notificación de registro exitoso}.
		
		\UCpaso 17. Despliega los clientes registrados, ordenados por fecha de registro vía la  \IUref{IU01}{Pantalla de listado de clientes registrados en el sistema}.
			
	\end{UCtrayectoria}
		
		\begin{UCtrayectoriaA}{A}{La recepcionista intenta registrar un cliente que no cumple con las precondiciones}
		
			\UCpaso Muestra el Mensaje {\bf MSG1-}``El aspirante a cliente con el CURP [{\em CURP del aspirante a cliente}] no puede ser registrado en el sistema.''.
			\UCpaso[\UCactor] Oprime el botón \IUbutton{Aceptar}.
			\UCpaso[] Termina el caso de uso.
		\end{UCtrayectoriaA}
		
		\begin{UCtrayectoriaA}{B}{Los datos del aspirante a cliente no coinciden con los extraídos del CURP}
			\UCpaso Muestra el Mensaje {\bf MSG2-}``Los datos del aspirante a cliente con el CURP [{\em CURP del aspirante a cliente}] no coinciden con el CURP introducido.''.
			\UCpaso[\UCactor] Oprime el botón \IUbutton{Aceptar}.
			\UCpaso[] Termina el caso de uso.
		\end{UCtrayectoriaA}
		
		
		\begin{UCtrayectoriaA}{C}{El CURP introducido no es válido}
			\UCpaso Muestra el Mensaje {\bf MSG3-}``El CURP [{\em CURP del aspirante a cliente}] no es válido.''.
			\UCpaso[\UCactor] Oprime el botón \IUbutton{Aceptar}.
			\UCpaso[] Termina el caso de uso.
		\end{UCtrayectoriaA}
		
		
		\begin{UCtrayectoriaA}{D}{La recepcionista abandona la operación}
			\UCpaso La recepcionista abandona la operación debido a que el cliente no puede continuar proporcionando sus datos.
			\UCpaso[\UCactor] Oprime el botón \IUbutton{Salir}.
			\UCpaso Cierra la sesión del usuario.
			\UCpaso Continua en el paso \ref{CU17Login} del \UCref{CU01}.
		\end{UCtrayectoriaA}		
		
		
		\begin{UCtrayectoriaA}{E}{El El RFC no es congruente con el CURP}
			\UCpaso Muestra el Mensaje {\bf MSG4-}``El RFC [{\em RFC del aspirante a cliente}] no es congruente con el CURP [{\em CURP del aspirante a cliente}].''.
			\UCpaso[\UCactor] Oprime el botón \IUbutton{Aceptar}.
			\UCpaso[] Termina el caso de uso.
		\end{UCtrayectoriaA}
		
		
		\begin{UCtrayectoriaA}{F}{El teléfono introducido no es válido}
			\UCpaso Muestra el Mensaje {\bf MSG5-}``El teléfono [{\em teléfono del aspirante a cliente}] no es válido.''.
			\UCpaso[\UCactor] Oprime el botón \IUbutton{Aceptar}.
			\UCpaso[] Termina el caso de uso.
		\end{UCtrayectoriaA}
		
		
		\begin{UCtrayectoriaA}{G}{El correo introducido no es válido}
			\UCpaso Muestra el Mensaje {\bf MSG6-}``El correo [{\em correo del aspirante a cliente}] no es válido.''.
			\UCpaso[\UCactor] Oprime el botón \IUbutton{Aceptar}.
			\UCpaso[] Termina el caso de uso.
		\end{UCtrayectoriaA}
		

		
%-------------------------------------- TERMINA descripción del caso de uso.